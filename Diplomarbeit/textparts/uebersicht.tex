\chapter{�bersicht}

	\section{Was ist Steganographie}
	
	Die Steganographie ist eine Wissenschaft die sich mit dem verstecken von zu �bermittelnden Nachrichten besch�ftigt. Das Wort kommt aus dem griechischen und bedeutet �bersetzt ''bedeckt schreiben'' oder auch ''geheimes Schreiben''. Sie kam schon in der Antike zum Einsatz und wurde vor allem dann verwendet, wenn selbst das �bertragen einer verschl�sselten Nachricht gef�hrlich werden konnte. 
	
	Zum Beispiel wenn ein Spion eine Nachricht an seinen Auftraggeber sendet kann er diese nicht einfach verschl�sseln. Ganz nach dem Satz ''Wer nichts zu verbergen hat braucht auch nicht verschl�sseln'' k�nnte jemand auf die Idee kommen das hier vertrauliche Informationen weitergegeben werden. Das ist der Moment wo man von Steganographie Gebrauch macht. Dabei wird n�mlich die eigentliche Nachricht, ob jetzt verschl�sselt oder nicht, in einer weiteren unauff�lligen Nachricht versteckt. F�r einen dritten ist nun nicht mehr ersichtlich ob sich Tats�chlich weitere Informationen in dem Tr�gertext vorhanden sind. 
	
	\cite{S: StegoGeschichte}
	
	\section{Einsatzgebiete}
	
	\section{Vor- und Nachteile}
	
	\section{Abgrenzung zur Kryptographie}
	
	\section{Steganographie als ''Wicked Problem''}